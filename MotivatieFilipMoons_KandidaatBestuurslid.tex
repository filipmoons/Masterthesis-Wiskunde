\documentclass[10pt,a4paper]{article}
\usepackage[dutch]{babel}
\usepackage{amsmath,amsthm}
\usepackage{amssymb}
\usepackage{marvosym}
\pagestyle{headings}
\usepackage{lmodern}
\usepackage{a4wide}
\usepackage{graphicx}
\usepackage{tikz}

\usepackage{pgf}
\usetikzlibrary{arrows}


\usepackage[font=small,format=plain,labelfont=bf,up,textfont=it,up]{caption}
\newcommand{\trace}{\operatorname{tr}}
\makeatletter
\usetikzlibrary{arrows}
\def\cleardoublepage{\clearpage\if@twoside \ifodd\c@page\else
\hbox{}
\vspace*{\fill}
\vspace{\fill}
\thispagestyle{empty}
\newpage
\if@twocolumn\hbox{}\newpage\fi\fi\fi}
\makeatother
\usepackage{makeidx}
\usepackage{framed}
\newenvironment{opmerking}[1]{\framed \rule{0ex}{0ex}\hspace{\stretch{1}}\textbf{#1}\hspace{\stretch{1}}\rule{0ex}{0ex}\\}{\endframed}


% THEOREMS -------------------------------------------------------
\newtheorem{thm}{Stelling}[section]
\newtheorem{cor}[thm]{Corollary}
\newtheorem{lem}[thm]{Lemma}
\newtheorem{prop}[thm]{Proposition}
\theoremstyle{definition}
\newtheorem{defn}[thm]{Definition}
\theoremstyle{remark}
\newtheorem{rem}[thm]{Remark}
\setcounter{equation}{0}
% MATH -----------------------------------------------------------
\newcommand{\norm}[1]{\left\Vert#1\right\Vert}
\newcommand{\abs}[1]{\left\vert#1\right\vert}
\newcommand{\set}[1]{\left\{#1\right\}}
\newcommand{\Real}{\mathbb R}
\renewcommand{\vec}[1]{\underline{#1}}
\newcommand{\voortbrengend}{\operatorname{vect}}
\newcommand{\spectrum} {\operatorname{spec}}
\newcommand{\eps}{\varepsilon}
\newcommand{\To}{\longrightarrow}
\newcommand{\BX}{\mathbf{B}(X)}
\newcommand{\A}{\mathcal{A}}

% ----------------------------------------------------------------
\begin{document}

\title{Kandidaatstelling bestuurschap LVSV Brussel\\Kandidaat bestuurslid}
\author{Filip Moons\\fmoons@vub.ac.be}
\date{2013-2014}
\maketitle

\section*{Persoonlijke achtergrond}
Ik ben Filip Moons, 22 jaar en studeer Wiskunde \& Computerwetenschappen aan de Vrije Universiteit
Brussel. Ik plan volgend jaar naar de 1ste Master Wiskunde (afstudeerrichting onderwijs) en de 1ste Master Toegepaste Informatica te gaan. Naast studeren ben ik erg actief als jeugdwerker bij Top Vakantie vzw, als secretaris bij Euro-Music vzw, internetco\"{o}rdinator bij Eurosong.be, lid van de wiskundevakgroep aan de VUB en als vormer over onderwijstechnologie bij (toekomstige) wiskundeleraars. Afgelopen jaar was ook de secretariaatspost van onze afdeling een grote uitlaatklep. Vanaf volgend academiejaar ben ik contractueel verbonden aan de uitgeverij die Keure; als hoofdredacteur van Vector, een gratis onderwijsblad voor alle Vlaamse wiskundeleraars. Mijn passie \& dromen liggen zonder twijfel in het onderwijs en onderwijsontwikkeling vooral met betrekking tot het vak Wiskunde. Daarom zou ik na mijn studies aan de VUB graag nog een jaar naar Londen gaan om er wiskundedidactiek te studeren.

\section*{Ideologische visie}
In het brede liberale spectrum situeer ik mezelf tussen het links-liberalisme en het klassiek liberalisme. Een overheid die voortdurend predikt het beter te weten dan de `gewone man', is een overheid die ik graag verkleind zie. Zo'n betuttelingen zijn immers een fundamentele en funeste aantasting van de persoonlijke vrijheid en de individuele mogelijkheden. De pretentie van `wij weten het beter' waarmee heelwat politici een hoop wetgeving doorvoeren, is afschrikwekkend. Als liberaal maak ik mij ook grote zorgen over het steeds groeiende Europese wetgevingsapparaat, het corporatisme dat zich rond ons heen voortdurend voltrekt en de economische crisis die we met nog meer regulering en nog meer overheidsschuld dreigen te beantwoorden. Toch wil ik mijn ogen niet sluiten voor de werkelijkheid: waar mensen samen leven, zijn regels en afspraken nodig, ook over zaken die we eventueel wel zien zitten om `collectief' te organiseren. In een voldoende vrije markt is er in mijn ogen wel ruimte voor publiek georganiseerd onderwijs en gezondheidszorg, al besef ik dat een publieke organisatie van deze zaken een moeilijke evenwichtsoefening zijn tussen budgettaire realiteit, collectieve wenselijkheid en organisatorische effici\"{e}ntie.

\section*{Het LVSV \& ik}
Ik ben reeds twee jaar actief bij onze afdeling. Eerst als bestuurslid onder voorzitter Dieter Keuten en afgelopen academiejaar als secretaris onder voorzitter Caroline Sneyers. Als secretaris beperkte ik mij niet tot het schrijven van verslagen, maar presenteerde ik een nieuwe website, deed ik een stevige duit in het organisatorisch zakje van ons lustrumgalabal, organiseerde ik het erg succesvolle euthanasiedebat - onze allereerste activiteit in Jette - en was ik tevens reisverantwoordelijke voor Brussel voor de nationaalreis naar Madrid. Er zijn nog allerlei hand-en-spandiensten waar de vereniging en haar bestuur steeds bij mij mee terecht konden. Ik heb tevens regelmatig contact gehad met bestuursleden van andere LVSV afdelingen. Dit alles maakt dat ik in alle bescheidenheid durf te stellen dat ik het LVSV en haar werking redelijk goed ken.

\section*{Kandidaatstelling bestuurslid}
Door op te komen als bestuurslid zet ik om persoonlijke redenen een bewuste stap terug bij onze vereniging, een vereniging waar ik met veel plezier een jaar de secretarisfunctie heb opgenomen. Mensen die benieuwd zijn naar mijn persoonlijke redenen: de profetische woorden \emph{``Some people dream of success while others wake up and work hard at it''} werden ooit door Winston Churchill uitgesproken, wel hard werken aan mijn levensdromen door de focus een beetje te herleggen en meer in te zoomen, dat is wat ik van plan ben volgend jaar. Het huidig (kern)bestuur kan overigens nog steeds op mijn positieve inbreng rekenen. Ik heb me ten opzichte van het nieuwe kernbestuur ge\"{e}ngageerd om opnieuw een succesvolle activiteit in Jette te organiseren. Ik zal tevens de taak van ledenverantwoordelijke op mij nemen, waarbij ik verantwoordelijk ben voor de administratieve registratie en opvolging van het ledenbestand van onze afdeling.\\

\noindent Ik wens u alvast te bedanken dit schrijven doorgenomen te hebben en reken op uw steun tijdens de algemene vergadering,\\

\noindent Hoogachtend,\\

\noindent Filip Moons
\end{document}


% ----------------------------------------------------------------
