% $Id: $
\documentclass[a4paper,11pt]{article}
\usepackage{a4wide}
\usepackage{graphicx}
\usepackage{caption}
\usepackage{subcaption}
\usepackage[labelfont=bf]{caption}
\usepackage{enumerate}
\usepackage{amsmath,amsthm,amssymb}
\usepackage{amsfonts}
% The following makes latex use nicer postscript fonts.
\usepackage{times}
\usepackage{subcaption}
\usepackage{datatool}
\usepackage[utf8]{inputenc}
\usepackage{pdflscape}
\usepackage{longtable}
\usepackage{tocloft}
\usepackage{epigraph}

\usepackage[dutch]{babel}
\usepackage{tikz}
\usepackage[toc,page]{appendix}

%\usepackage[colorlinks,urlcolor=blue,linkcolor=blue]{hyperref}
\pagestyle{headings}
\newcommand{\upuparrow}{\mathrel{\reflectbox{\rotatebox[origin=c]{90}{$\twoheadrightarrow$}}}}
\newcommand{\downdownarrow}{\mathrel{\reflectbox{\rotatebox[origin=c]{90}{$\twoheadleftarrow$}}}}
\usepackage{vubtitlepage}
\usepackage{lmodern}
\usepackage{graphicx}

\usepackage[geometry]{ifsym}
%\usepackage[font=small,format=plain,labelfont=bf,up,textfont=it,up]{caption}
\renewcommand{\thefigure}{\thesection.\arabic{figure}}
\author{Filip Moons}
\title{Werkstuk}

\newtheorem{theorem}{Theorem}[section]
\newtheorem{lemma}[theorem]{Lemma}
\newtheorem{proposition}[theorem]{Proposition}
\newtheorem{conjecture}{Conjecture}
\newcommand{\tussen}[1]{\paragraph*{#1}\mbox{}\\}
\newtheorem{property}[theorem]{Property}
\newtheorem{definition}[theorem]{Definition}
\newtheorem{corollary}[theorem]{Corollary}
\newtheorem{remark}[theorem]{Remark}
\newtheorem{remarks}[theorem]{Remarks}
\newtheorem{notation}[theorem]{Notation}
\theoremstyle{definition}
\newtheorem{example}[theorem]{Example}
\newtheorem{examples}[theorem]{Examples}
 \usepackage[table,xcdraw]{xcolor}
\setcounter{tocdepth}{5}
\newcommand{\N}{{\mathbb N}}
\newcommand{\Z}{{\mathbb Z}}
\newcommand{\Q}{{\mathbb Q}}
\newcommand{\R}{{\mathbb R}}
\newcommand{\C}{{\mathbb C}}
\newcommand{\HQ}{{\mathbb H}}
\renewcommand{\P}{{\mathbb P}}
\newcommand{\E}{{\mathbb E}}
\newcommand{\cost}{\text{cost}}
\newcommand{\Nash}{\text{Nash}}
\newcommand{\Tau}{\mathrm{\tau}}
\newcommand{\nash}{\text{nash}}
\newcommand{\opt}{\text{opt}}
\newcommand{\LFP}{\text{LFP}}
\renewcommand{\int}{\text{int}}
\newcommand{\enquote}[1]{`#1'}
%\newenvironment{proof}{\noindent{\bf Bewijs.}}{{\hfill $ \ Box $}\vskip 4mm}

\promotortitle{Titularis}
\promotor{Prof. Dr. N. Engels\\}
\advisors{}
\advisortitle{}
\begeleider{}
\addto\captionsenglish{\renewcommand*\abstractname{Abstract for non-mathematicians}}
\date{MEI 2006}
\faculty{Specifieke Lerarenopleiding}
\advisortitle{}
\department{Wetenschappen \& Ingenieurswetenschappen}
\reason{Pedagogische Vraagstukken}

\date{Juni 2015}


\begin{document}
% Then english TitlePage
\maketitlepage


\tableofcontents
\newpage
\section{Inleiding}
Dit stageverslag bespreekt alle facetten van de Verbredende oefenstage, die ik heb doorlopen op de Heemschool in februari 2015. 
Het stagerapport bevat mijn motiviatiebrief, bespreekt de stageschool, de observaties en bevat een zeer grondige toelichting van het uitgevoerde project. 
Daarnaast vindt u ook nog een logboek terug en concluderen we het rapport met enkele kritische beschouwingen. De bibliografie en een afdruk
van de stageovereenkomst sluiten het geheel af. Het verslag 
probeert de lezer een zo goed mogelijk beeld te geven over het doel en de 
werkzaamheden tijdens de stage.
\newpage
\section{Wie ben ik?}
Ik ben Filip Moons, 24 jaar en momenteel combineer ik de opleidingen master Wiskunde, master Toegepaste Informatica
en de specifieke lerarenopleiding Wetenschappen \& Ingenieurswetenschappen. Ik heb namelijk een passie voor
Wiskunde, Informatica en het onderwijs. Mijn passie voor het onderwijs ontdekte ik tijdens mijn engagement bij het jeugdwerk waarbij ik zogoed als
elke schoolvakantie mee  kampen begeleid bij Top Vakantie vzw. In juni 2015 studeer ik af in alle opleidingen. 

\section{Werkverslagen}

\section{Algemene visie over mijn rol als `leraar als opvoeder'}
 \end{document}

