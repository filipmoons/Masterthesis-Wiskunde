\documentclass{beamer}
%\documentclass[trans]{beamer} %om te printen!
%\transglitter etc: kunt ge pas zien op Full Screen (Ctrl+L)
\usepackage{etex}
\usepackage{tikz}
\usepackage{amsmath,amsthm,amssymb}
\usepackage{amsfonts}
\usepackage{graphicx,multicol}

\usepackage[all]{xy}
\usepackage{enumerate}

\usepackage{graphicx}
\usepackage{beamerouterthememiniframes, beamercolorthemeann,srcltx,hyperref}
\usepackage{amsmath,amsthm,amssymb}

\newcommand{\upuparrow}{\mathrel{\reflectbox{\rotatebox[origin=c]{90}{$\twoheadrightarrow$}}}}
\newcommand{\downdownarrow}{\mathrel{\reflectbox{\rotatebox[origin=c]{90}{$\twoheadleftarrow$}}}}
\setbeamercolor{normal text}{fg=black!70}
\setbeamertemplate{navigation symbols}{}%geen navigatie
\setbeamertemplate{blocks}[rounded][shadow=true]

%\setbeamercovered{dynamic} %te komen items in lichtgrijs
%\setbeamercolor{background canvas}{bg=black!0}%is wit

\logo{\vspace{-0.2cm}\\
\hfill\includegraphics[height=3cm]{VUB_schild}}
%plaatst het logo op elke slide onderaan

\newcommand{\Z}{\mathbb{Z}}
\newcommand{\Q}{\mathbb{Q}}
\renewcommand{\P}{\mathbb{P}}
\newcommand{\R}{\mathbb{R}}
\newcommand{\N}{\mathbb{N}}
\newcommand{\C}{\mathbb{C}}
\newcommand{\U}{\mathcal{U}}
\newcommand{\E}{\mathbb{E}}
\newcommand{\z}{\mathcal{Z}}
\newtheorem{remark1}{Remark}
\newcommand{\LFP}{\text{LFP}}
%\columnsep=1.8cm %\columnseprule=.4pt
\newcommand{\cost}{\text{cost}}
\newcommand{\Nash}{\text{Nash}}
\newcommand{\nash}{\text{nash}}
\newcommand{\opt}{\text{opt}}
\newcommand{\copt}{\cost(a_{\opt})}
\title{Rate-Monotonic Scheduling}
\author{Filip Moons\\Master in Applied Computer Science\\Promotor: Prof. Dr. Martin Timmerman\\Presentation Operating Systems \& Security}
\date{Friday 9 January, 2014}

\begin{document}

\begin{frame}[plain]
\includegraphics[width=0.4\paperwidth]{VUB_logo.jpg}
\vspace{2cm}
\titlepage
\end{frame}


%\section[Overzicht]{}%rechte haken dienen om niet in de outline te komen
%maar wel vanboven in het donkergroene balkje

%\begin{frame}[plain]{Outline}
%\end{frame}
%
%\begin{frame}[plain]{Outline}
%\tableofcontents[pausesections]
%\end{frame}
%
%
%\section{blabla}
%
%\begin{frame}
%\begin{alertblock}{}
%\begin{center}
%blabla
%\end{center}
%\end{alertblock}
%\end{frame}

\begin{frame}{Content}
\begin{itemize}
  \item Tasks
  \item The algorithm
  \item Tests
  \item Example
  \end{itemize}

\end{frame}



\section{Tasks}
\begin{frame}{Some assumptions}
\begin{block}{Definition}
A task $\tau_i$ is a process that has:
\begin{itemize}
        \item To be periodically executed in a period $T_i$       
         \item The worst case execution time $C_i$
        \item A deadline $D_i$, which is the available time on the processor. $D_i = T_i$
        \end{itemize}
\end{block}
    
\end{frame}

\begin{frame}{Some assumptions}
\begin{block}{Definition}
A task $\tau_i$ is a process that has:
\begin{itemize}
        \item To be periodically executed in a period $T_i$       
         \item The worst case execution time $C_i$
        \item A deadline $D_i$, which is the available time on the processor. $D_i = T_i$
        \end{itemize}
\end{block}
    
\end{frame}


%\begin{frame}
%\begin{block}{}
%\begin{center}
%Introductie
%\end{center}
%\end{block}
%\end{frame}
\section{The algorithm}

\begin{frame}{The algorithm}


\begin{block}{Rate-Monotonic Scheduling: Algorithm}
\begin{enumerate}
  \item The task with the smallest period has the highest priority,
  \item A higher-priority task ready to be executed, overrides the current 
  executed task. The current executed task is is interrupted and may resume 
  afterwards.
\end{enumerate}\end{block}

\end{frame}


\section{Schedulability tests}
\begin{frame}{An example}
\begin{block}{}
  \begin{enumerate}
    \item $\tau_1$: $T_1$ = 4 ms,  $C_1$ = 1 ms 
        \item $\tau_2$: $T_2$ = 5 ms,  $C_2$ = 2 ms
             \item  $\tau_3$: $T_3$ = 7 ms,   $C_1$ = 2 ms 
  \end{enumerate}
\end{block}
\end{frame}

\begin{frame}{Schedulability test 1}
\begin{block}{The utilization factor}
  The utilization factor of a task set $\tau_1, \tau_2, ... , \tau_n$ is:
$$ U = \sum\limits_{i=1}^{n} \frac{C_i}{T_i}$$

$\frac{C_i}{T_i}$ gives the utilization of task $\tau_i$ on the CPU
\end{block}

\begin{block}{Schedulability test 1: Liu \& Layland lower bound}
  With $n$-tasks with, a schedule exists as:
        $$U \leq n(2^{\frac{1}{n}} -1)$$
 For $n \to \infty$, we get: $\lim_{n \rightarrow \infty}   n(2^{\frac{1}{n}} -1) = \ln 2 \approx 
0.693147...$
\end{block}


\end{frame}

\begin{frame}{Schedulability test 2}

\begin{block}{The RT-test}
 A task set can be scheduled by RMS if the deadline of the first execution of each task is met when using the scheduling algorithm starting 
 all tasks at the same time.
\end{block}

\begin{block}{Total processing requirement}
The total processing requirement $u_i(t)$ of a task $\tau_i$ in the time interval $[0,t]$ 
is given by, with $0<t \leq T_i$:
\begin{eqnarray}
 u_i(t) = C_i + \sum\limits_{k=1}^{i-1} \left \lceil{\frac{t}{T_i}}\right \rceil C_k
 \end{eqnarray}
The idea is immediately clear: if $u_i(t) \leq t$ for some $t \leq T_i$ then task $\tau_i$
is schedulable. 
\end{block}
\end{frame}

\section{Example}

\begin{frame}{An example}
\begin{block}{}
  \begin{enumerate}
    \item $\tau_1$: $T_1$ = 4 ms,  $C_1$ = 1 ms $\to \frac{C_1}{T_1} = 0.25$,
    \item $\tau_2$: $T_2$ = 5 ms,  $C_2$ = 2 ms $\to \frac{C_2}{T_2} = 0.4$,
     \item  $\tau_3$: $T_3$ = 7 ms,   $C_1$ = 2 ms $\to \frac{C_2}{T_2} = 0.28$,
  \end{enumerate}
\end{block}
\begin{block}{Schedulability test 1}
  
With $n=3$, the sum of $ \frac{C_i}{T_i}$ must be lower than 0.7798.  We become that 
  $0.25 + 0.4 + 0.28 = 0, 91 > 0.7798$. 
  \end{block}
\end{frame}

\begin{frame}{An example}

\begin{block}{Schedulability test 2}
 \begin{enumerate}
  \item $u_1(t) = C_1 = 1 \leadsto u_1(4) = 1 $
   \item $u_2(t) = ..  \leadsto u_2(4) = 3, u_2(5) = 4$
   \item $u_3(t) = ... \leadsto u_3(4) = 5, u_3(5) = 6, u_3(7) = 8$
     \end{enumerate}
We test:
\begin{enumerate}

  \item $u_1(t) \leq t$ satisfied for $t=4? \leadsto u_1(4) = 1 \leq 4 \Rightarrow$ \textbf{OK!}

  
  \item $u_2(t) \leq t$ satisfied for $t\in{4,5}? \leadsto u_2(4) \leq 4,  u_2(5) \leq 5 \Rightarrow$ \textbf{OK!}
  \item $u_3(t) \leq t$ satisfied for $t\in{4,5,7}? \leadsto u_3(4) > 4,  u_3(5) > 5, u_3(7) > 7 \Rightarrow$ \textbf{\underline{NOT} OK!}
\end{enumerate}


  
  \end{block}
\end{frame}



\end{document}
