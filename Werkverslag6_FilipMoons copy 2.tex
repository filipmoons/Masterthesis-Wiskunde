\documentclass[10pt,a4paper]{article}
\usepackage[dutch]{babel}
\usepackage{amsmath,amsthm}
\usepackage{amssymb}
\usepackage{marvosym}
\pagestyle{headings}
\usepackage[usenames,dvipsnames]{color}


\usepackage{lmodern}
\usepackage{a4wide}
\usepackage{soul}
\usepackage{graphicx}
\usepackage{tikz}
\usepackage[utf8]{inputenc}
\usepackage[T1]{fontenc}
\usepackage{lmodern} % load a font with all the characters
\usepackage{pgf}
\usepackage[options]{natbib}
\usetikzlibrary{arrows}


% THEOREMS -------------------------------------------------------
\newtheorem{thm}{Stelling}[section]
\newtheorem{cor}[thm]{Corollary}
\newtheorem{lem}[thm]{Lemma}
\newtheorem{prop}[thm]{Proposition}
\theoremstyle{definition}
\newtheorem{defn}[thm]{Definition}
\theoremstyle{remark}
\newtheorem{rem}[thm]{Remark}
\setcounter{equation}{0}
% MATH -----------------------------------------------------------
\newcommand{\norm}[1]{\left\Vert#1\right\Vert}
\newcommand{\abs}[1]{\left\vert#1\right\vert}
\newcommand{\set}[1]{\left\{#1\right\}}
\newcommand{\Real}{\mathbb R}
\renewcommand{\vec}[1]{\underline{#1}}
\newcommand{\voortbrengend}{\operatorname{vect}}
\newcommand{\spectrum} {\operatorname{spec}}
\newcommand{\eps}{\varepsilon}
\newcommand{\To}{\longrightarrow}
\newcommand{\BX}{\mathbf{B}(X)}
\newcommand{\A}{\mathcal{A}}

% ----------------------------------------------------------------
\begin{document}

\title{Nachtmerrie in de stagecontext - Dinsdag 25 november 2014 \\Reflecterend \& Onderzoekend Handelen}
\author{Filip Moons\\Specifieke Lerarenopleiding Wetenschappen \& Ingenieurswetenschappen (Wiskunde)}
\maketitle
\section{\emph{Hoe artificieel is een stagecontext?}}

\subsection{Situatieschets}
De stage zit erop. De pralines zijn aan de mentoren uitgedeeld. De laatste taken en toetsen zijn verbeterd. Tijd voor een terugkeer naar een stageloos, normaal leven. Een pak ervaringen rijker & enkele illusies armer trek ik de deuren van de schoolpoort even achter mij dicht, tot september, denk ik. Ook enkele vragen: 'Wat moest ik hiervan denken?' neem ik mee. Zo had ik bijvoorbeeld een bijzonder moeilijk beheersbare klas en had ik daar moeite met klasmanagement, terwijl het in de andere klassen werkelijk vlekkeloos verliep.

\subsection{Interpretatie van de feiten}
Tijdens de stage gebeuren allerlei zaken die inherent zijn aan de stagecontext en die in een werkelijke situatie zich nauwelijks zouden voordoen. Neem nu die moeilijke klas met mijn lamentabel klasmanagement: de desbetreffende mentor had tijdens mijn observatieuurtjes zelf de grootste moeite met die klas een beetje onder controle te houden. Hoe zou ik als weerloze stagiair het ooit beter kunnen doen? Ik ben er ergens zeker van dat die klas wel beheersbaar is als je de échte leraar bent die van in het begin voor die klas stond. Daarom vraag ik mij soms af in welke mate ik mezelf moet in vraag stellen want zijn die ervaringen niet inherent aan de stagecontext? Zouden deze problemen zich werkelijk voordoen moest ik de leraar geweest zijn? Heb ik in deze niet gewoon de klasmanagementproblemen van de mentor overgeërfd?  
\subsection{Leervragen}
\begin{itemize}
  \item Hoe kan ik ervaringen opdelen in twee categorieën: een categorie 'uitsluitend voorkomend tijdens de stage' en een categorie 'kan ook in werkelijkheid voorkomen'? Om zo te kunnen bepalen welke ervaringen de moeite waard zijn om lang bij stil te staan, en welke niet?    
 \end{itemize}

 

\end{document}


% ----------------------------------------------------------------
